All the general formulae below make use of ``esoteric'' level mathematical operations and functions.

\subsection{Creating \emph{e}}
Throughout this section, we will use the combination of $\ln(x)$ and $e$ to create numbers.
$e$ can be created from each of the four digits with the function $\exp(x)=e^x$.
$\exp(1)=e$, so any digit can be first transformed to 1 and then to $e$.

Alternatively, if exp is considered cheating, the complex function $\cis(\theta)=\cos(\theta)+i\sin(\theta)$ along with Euler's identity can be used. 
Once again, any digit can be transformed to 1, then i, then $\cis(-i)=e^{i(-i)}=e$.

\subsection{Integers}

The general solution for we present and extend below is adapted from Dirac (1936).

It is inspired by the famous four fours problem, which states that any positive whole number can be created with just four 4s, relying on the use of square roots:

\begin{equation}
  n = -\log_\frac{\sqrt{4}}{4}\left( \log_4 \underbrace{\sqrt{\sqrt{\cdots\sqrt{\sqrt{4}}}}}_{n} \right)
\end{equation}

where $n=\textrm{number of radicals}$.
We can extend this to the general case by recognizing that the only requirement is that the base of the outer logarithm has to be two and the base of the inner logarithm has to be the same as the radicand.
We can prove this:

\begin{align*}
  n &= -\log_2\left(\log_a \underbrace{\sqrt{\sqrt{\cdots\sqrt{\sqrt{a}}}}}_{k} \right) \\
  &= -\log_2\left(\log_a \left(a^{0.5^k}\right)\right) \\
  &= -\log_2\left(\frac{1}{2^k}\right) \\
  &= -(-k) \\
  &= k \\
\end{align*}

We can further simplify this expression by switching into base $e$.
We can use this to write any integer with only two numbers:
\begin{align*}
  n &= -\log_2\left( \ln \sqrt{\cdots\sqrt{e}}\right) \\
  &= -\log_2\left( \ln \sqrt{\cdots\sqrt{\exp1}}\right)
\end{align*}

This formula holds for all positive integers $n\in\mathbb{Z}^+$.
Zero can be created with no square roots.
By removing the leading negative, we can create all integers $n\in\mathbb{Z}$.
To enforce the use of all four digits, simply add a $+0^9$ term to the end.

\begin{equation}
  n = -\log_2\left(\ln \sqrt{\cdots\sqrt{\exp1}}\right)+0^9
\end{equation}

The operational complexity for this formula is $l=n+6$ for $n>0$ and $l=n+5$ for $n\leq0$.

\subsection{Rationals}
Since we can represent any integer with two numbers, we can represent any rational number with two sets of two numbers.
For any $n=\frac{a}{b}$, where $n\in\mathbb{Q}$:
\begin{align*}
  n &= \frac{a}{b} \\
  &= \frac{
      \log_2\left(
        \ln \underbrace{\sqrt{\cdots\sqrt{\exp1}}}_{a} 
      \right)
    }{
      \log_2\left(
        \ln \underbrace{\sqrt{\cdots\sqrt{\exp1}}}_{b} 
      \right)
    } \\
  &= \log_{
      \ln \underbrace{\sqrt{\cdots\sqrt{\exp1}}}_{b}
    }
    \left(
      \ln \underbrace{\sqrt{\cdots\sqrt{\exp1}}}_{a}
    \right)
\end{align*}

Once again, applying to 2019:
\begin{equation}
  n = \log_{
      \ln\sqrt{\cdots\sqrt{\exp(2^0)}}
    }
    \left(
      \ln\sqrt{\cdots\sqrt{\exp(1^9)}}
    \right)
\end{equation}

Following the same logic as for integers, a negative sign can be prepended for negative rationals.
Thus, the operational complexity is $l=7+a+b$ for $n\geq0$ and $l=8+a+b$ for $n<0$.

\subsection{Complex}
\subsubsection{Gaussian Integers}
Very similar to how we created the set of rationals, all Gaussian integers $\mathbb{Z}[i]=\{a+bi\mid a,b\in\mathbb{Z}\}$ can created as the sum of integers. For any $n\in\mathbb{Z}[i]$:
\begin{align*}
  n &= a + bi \\
  &= -\log_2\left(
      \ln \underbrace{\sqrt{\cdots\sqrt{\exp1}}}_{a} 
    \right) -
    \log_2\left(
      \ln \underbrace{\sqrt{\cdots\sqrt{\exp1}}}_{b} 
    \right)i \\
  &= -\log_2\left(
      \ln \sqrt{\cdots\sqrt{\exp1}}
      \times
      \ln^i \left(\sqrt{\cdots\sqrt{\exp1}}\right)
    \right) \\
  &= -\log_2\left(
      \ln \sqrt{\cdots\sqrt{\exp1}}
      \times
      \ln^{\sqrt{-1}} \left(\sqrt{\cdots\sqrt{\exp1}}\right)
    \right)
\end{align*}

Applying to 2019:
\begin{equation}
  n = -\log_2\left(
    \ln \sqrt{\cdots\sqrt{\exp0!}}
    \times
    \ln^{\sqrt{-1}} \left(
      \sqrt{\cdots\sqrt{\exp .\overline{9}}}
    \right)
  \right)
\end{equation}

The above equation has a operational complexity of $l=13+a+b$.
This works for all Gaussian integers with positive $a$ and $b$.
Apply the complex conjugate operator $\bar{n}$ to create $a-bi$ ($l=14+a+b$).
Remove the leading negative to create numbers of the form $-a-bi$ ($l=12+a+b$).
Finally, for those of the form $-a+bi$, do both ($l=13+a+b$).

\subsubsection{Gaussian Rationals}
Real rationals, like real integers, can be created with just two numbers.
From this, we can create the set of Gaussian rationals $\mathbb{Q}[i]=\{\frac{a}{b}\mid a,b\in\mathbb{Z}[i]\}$.
% TODO: prove this
Every $n\in\mathbb{Q}[i]$ can be expressed as the sum $q_1+q_2i$ where $q_1,q_2\in\mathbb{Q}$.
\begin{align*}
  n &= q_1+q_2i \\
  &= \frac{a_1}{b_1} + \frac{a_2}{b_2}i
\end{align*}
where $a_1, a_2, b_1, b_2\in\mathbb{Z}$.
It is important that our fractions contain the same denominator.
Let $d=\textrm{lcm}(b_1, b_2)$, $k_1=\frac{d}{a_1}$, $k_2=\frac{d}{a_2}$.
\begin{align*}
  &= \frac{k_1}{d} + \frac{k_2}{d}i \\
  &= \log_{
      \ln \underbrace{\sqrt{\cdots\sqrt{\exp1}}}_{d}
    }
    \left(
      \ln \underbrace{\sqrt{\cdots\sqrt{\exp1}}}_{k_1}
    \right) +
    \log_{
      \ln \underbrace{\sqrt{\cdots\sqrt{\exp1}}}_{d}
    }
    \left(
      \ln \underbrace{\sqrt{\cdots\sqrt{\exp1}}}_{k_2}
    \right)i \\
  &= \log_{
      \ln \underbrace{\sqrt{\cdots\sqrt{\exp1}}}_{d}
    }
    \left(
      \ln \underbrace{\sqrt{\cdots\sqrt{\exp1}}}_{k_1}
      \times
      \ln^{i} \underbrace{\sqrt{\cdots\sqrt{\exp1}}}_{k_2}
    \right)
\end{align*}

Substituting in the digits of 2019:
\begin{equation}
  n = \log_{\ln \sqrt{\cdots\sqrt{\exp\Gamma(2)}}}
    \left(
      \ln \sqrt{\cdots\sqrt{\exp\sqrt{0!}}}
      \times
      \ln^{\sqrt{-1}} \sqrt{\cdots\sqrt{\exp .\overline{9}}}
    \right)
\end{equation}
There are a lot of square root chains here.
The total number of variadic square roots in this expression is
\begin{equation*}
  n_{sqrt}=d+k_1+k_2=\textrm{lcm}(b_1,b_2) * \left(1+\frac{1}{a_1}+\frac{1}{a_2}\right),
\end{equation*}
making the operational complexity $l=13+n_{sqrt}$.
Like the last equation, the three other sign combinations can be created with the complex conjugate operator and/or a leading negative.
