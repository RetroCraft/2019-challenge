\documentclass{article}

% Format things
\usepackage{longtable}
\usepackage{booktabs}
\usepackage{multicol}

% Math things
\usepackage{amsmath}
\usepackage{amssymb}
\usepackage{mathtools}
\usepackage{hyperref}
% ceil/floor, abs
\DeclarePairedDelimiter\ceil{\lceil}{\rceil}
\DeclarePairedDelimiter\floor{\lfloor}{\rfloor}
\DeclarePairedDelimiter\abs{\lvert}{\rvert}
% cis, exp
\DeclareMathOperator{\cis}{cis}
% redefine * as \times
\DeclareMathSymbol{*}{\mathbin}{symbols}{"02}

\title{2019 Challenge}
\author{James Ah Yong, QiLin Xue}
\date\today

\begin{document}

\maketitle

\section{Introduction}

The problem, as presented by Dwarka (2019), is as follows:

\begin{quotation}
  Consider the digits 2, 0, 1, and 9.
  Using each digit only once, create a formula which generates all positive integers from 1 to 100.
  Any mathematical operator is allowed, including concatenation of digits.
\end{quotation}

In this paper, we solve this problem with elementary operations and also provide a general solution using more advanced and esoteric mathematical operators.
For our purposes, an ``elementary'' operator is one of the basic infix operations: addition ($+$), subtraction ($-$), multiplication ($\times$), and division ($\div$).
Concatenation of digits (i.e.\ using 2 and 0 to form 20) and exponentiation (including radicals) are also allowed.

The ``advanced'' operations are those which would be familiar to high-school level math.
This includes: logarithms, basic and inverse trigonometric functions, and the probability functions ($x!$, $\binom{n}{r}$ or $_nC_r$, and $_nP_r$).

The ``esoteric'' level is all other operations that might aid in reducing the number of operations to find shortest solutions.
This contains functions like the gamma function ($\Gamma(x) = (x-1)!$) or the cis function ($\mathrm{cis}(x) = \cos(x) + i \sin(x)$).

Finally, ``cheaty'' functions are those that make some solutions feel like cheating.
Specifically, the floor ($\floor{x}$) and ceiling ($\ceil{x}$) functions. 

Every solution is assigned an operational complexity $l$.
This value is defined as the number of operations required to reach the solution.
For example, $\frac{a+b}{c}=(a+b)\div c$ has a complexity of 3.
Concatenation does not count as an operation, so $20+19$ has a complexity of 1.

\section{Specific Solutions}

The following are the shortest solutions we could find, measuring length by the number of operators (not including concatenation).
In the problem, all digits do not have to be used, however, we chose to use all digits for all problems.
For the table below, each equation equals $n$ and has $l$ operators.
Wherever possible, an effort is made to preserve the order of the digits as 2, 0, 1, 9 or its reverse.

Advanced and esoteric solutions are only shown if they are shorter or match the 2, 0, 1, 9 order where the elementary solution does not.

%\setlength\LTleft{-1in}
%\setlength\LTright{-1in}
\begin{longtable}{r l r l r l r}
\toprule
$n$ & Elementary & ($l$) & Advanced & ($l$) & Esoteric & ($l$) \\ \midrule
\endhead%
0 & $0*219$ & 1 \\ \midrule
1 & $20-19$ & 1 \\ \midrule
2 & $2+0^{19}$ & 2 \\ \midrule
3 & $12-09$ & 1 \\ \midrule
4 & $\sqrt{9}+1^{20}$ & 3 \\ \midrule
5 & $2+01+\sqrt{9}$ & 3 \\ \midrule
6 & $9-2-01$ & 2 \\ \midrule
7 & $9-2*01$ & 2 \\ \midrule
8 & $9-1^{02}$ & 2 \\ \midrule
9 & $9+0^{12}$ & 2 \\ \midrule
\midrule
10 & $9+1^{02}$ & 2 \\ \midrule
11 & $2^{01}+9$ & 2 \\ \midrule
12 & $21-09$ & 1 \\ \midrule
13 & $12+9^0$ & 2 \\ \midrule
14 & $12+9^0$ & 2 \\ \midrule
15 & $12+\sqrt{09}$ & 2 & $(2+01)!+9$ & 3 \\ \midrule
19 & $29-10$ & 1 \\ \midrule
\midrule
20 & & & $9*2+0!+1$ & 4 \\ \midrule
21 & $12+09$ & 1 \\ \midrule
22 & & & & & $\ceil{21.90}$ & 1 \\ \midrule
23 & & & & & $\ceil{201\div9}$ & 2 \\ \midrule
29 & $20^1+9$ & 2 & & & $\floor{29.10}$ & 1\\ \midrule
\midrule
30 & $\sqrt{10^2*9}$ & 3 & & & $\ceil{29.10}$ & 1\\ \midrule
33 & & & $(2+0!+1)!+9$ & 5 \\ \midrule
38 & $02*19$ & 1 \\ \midrule
39 & $20+19$ & 1 \\ \midrule
\midrule
40 & & & $2*(0!+19)$ & 3 \\ \midrule
47 & & & & & $\floor{\ln 201^9}$ & 3 \\ \midrule
48 & & & & & $\ceil{\ln 201^9}$ & 3 \\ \midrule
49 & & & ${(9-1-0!)}^2$ & 4 \\ \midrule
\midrule
51 & & & & & $\floor{2^9\div10}$ & 3 \\ \midrule
52 & & & & & $\ceil{2^9\div10}$ & 3 \\ \midrule
54 & & & $(2+01)!*9$ & 3 & $\floor{109\div2}$ & 2 \\ \midrule
55 & & & & & $\ceil{109\div2}$ & 2 \\ \midrule
58 & & & $29*(0!+1)$ & 3 & $\floor{\ln 19^{20}}$ & 3 \\ \midrule
59 & & & & & $\ceil{\ln 19^{20}}$ & 3 \\ \midrule
\midrule
63 & $21*\sqrt{09}$ & 2 \\ \midrule
69 & $90-21$ & 1 \\ \midrule\midrule
71 & $91-20$ & 1 \\ \midrule
78 & $90-12$ & 1 \\ \midrule
79 & & & $9^2-0!-1$ & 4 \\ \midrule
\midrule
80 & $20*(1+\sqrt{9})$ & 3 \\ \midrule
82 & $92-10$ & 1 \\ \midrule
87 & $90-1-2$ & 2 \\ \midrule
89 & $91-02$ & 1 \\ \midrule
\midrule
90 & & & $92-0!-1$ & 3 \\ \midrule
91 & $92-01$ & 1 \\ \midrule
92 & & & & & $\ceil{91.02}$ & 1 \\ \midrule
93 & $91+02$ & 1 \\ \midrule
93 & $92+01$ & 1 \\ \midrule
95 & $190\div2$ & 1 \\ \midrule
99 & & & $(12-0!)*9$ & 3 \\ \midrule
\midrule
100 & ${(9+1)}^{02}$ & 2 \\ \bottomrule
\end{longtable}

\section{General Formulae}

All the general formulae below make use of ``esoteric'' level mathematical operations and functions.

\subsection{Modifying The Original Numbers}

An important part of our discussion is the ability for $2$, $0$, $1$, and $9$, to be modified to become other numbers. Thus, from now on we will pretend our four numbers can be anything from $1$ to $4$, inclusive. For example, $0! = 1$, $\ceil{\arcsin(1)}=2$, $\floor{\cosh(2)}=3$, and $\ceil{\cosh(2)}=4$.

Thus, we are able to transform these four numbers around.

\subsection{Integers}

The general solution we present below is adapted from Wang (2019).
It is inspired by the famous four ``4''s  problem, which states that any real number can be created with just four 4s, relying on the use of square roots:

\begin{equation}
  n = -\log_\frac{\sqrt{4}}{4}\left( \log_4 \underbrace{\sqrt{\sqrt{\cdots\sqrt{\sqrt{4}}}}}_{n} \right)
\end{equation}

where $n=\textrm{number of radicals}$. We can extend this to the general case by recognizing that the only requirement is that the base of the outer logarithm has to be two and the base of the inner logarithm has to be the same as the radicand. We can prove this:

\begin{align*}
  n &= -\log_2\left(\log_a \underbrace{\sqrt{\sqrt{\cdots\sqrt{\sqrt{a}}}}}_{k} \right) \\
  &= -\log_2\left(\log_a \left(a^{0.5^k}\right)\right) \\
  &= -\log_2\left(\frac{1}{2^k}\right) \\
  &= -(-k) \\
  &= k \\
\end{align*}

To apply it to the specific $2019$ we can use the fact that $\arccos(0) = \arcsin(1)$ giving us the formula:
\begin{equation}
  n = -\log_2\left( \log_{\arccos(0)} \sqrt{\sqrt{\cdots\sqrt{\arcsin(1)}}}\right)
\end{equation}

We can further simplify this expression by switching into base $e$ and using the complex function:
\begin{equation}
  \mathrm{cis}(\theta) = \cos\theta + i\sin\theta = e^{i\theta}
\end{equation}

Therefore, $\mathrm{cis}(-i)=\mathrm{cis}(-\sqrt{-1})=e$. We can use this to write any integer with only two numbers:
\begin{equation}
  n = -\log_2\left( \ln \sqrt{\sqrt{\cdots\sqrt{\mathrm{cis}(\sqrt{-1})}}}\right)
\end{equation}

This formula holds for all positive integers $n\in\mathbb{Z}^+$.
Zero can be created with no square roots.
By removing the leading negative, we can create all integers $n\in\mathbb{Z}$.
To enforce the use of all four digits, simply add a $+0^9$ term to the end.

\begin{equation}
  n = -\log_2\left( \ln \sqrt{\sqrt{\cdots\sqrt{\mathrm{cis}(\sqrt{-1})}}}\right)+0^9
\end{equation}

The operational complexity (see definition above) for this formula is $l=n+8$ for $n>0$ and $l=n+7$ for $n\leq0$.

\subsection{Rationals}
Since we can represent any integer with two numbers, we can represent any rational number with two sets of two numbers.
For any rational number $n=\frac{a}{b}$:
\begin{align*}
  n &= \frac{
    \log_2\left(
      \ln \underbrace{\sqrt{\cdots\sqrt{\mathrm{cis}(\sqrt{-1})}}}_{a} 
    \right)
  }{
    \log_2\left(
      \ln \underbrace{\sqrt{\cdots\sqrt{\mathrm{cis}(\sqrt{-1})}}}_{b} 
    \right)
  } \\
  &= \log_{
    \ln \underbrace{\sqrt{\cdots\sqrt{\mathrm{cis}(\sqrt{-1})}}}_{b}
  }
  \left(
    \ln \underbrace{\sqrt{\cdots\sqrt{\mathrm{cis}(\sqrt{-1})}}}_{a}
  \right)
\end{align*}

Once again, applying to 2019:
\begin{equation}
  n = \log_{
    \ln\sqrt{\cdots\sqrt{\mathrm{cis}(\sqrt{-2^0})}}
  }
  \left(
    \ln\sqrt{\cdots\sqrt{\mathrm{cis}(\sqrt{-1^9})}}
  \right)
\end{equation}

Following the same logic as for integers, a negative sign can be prepended for negative rationals. The operational complexity is $l=11+a+b$ for $n\geq0$ and $l=12+a+b$ for $n<0$.

\subsection{Complex}


\end{document}