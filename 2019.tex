\documentclass{article}

% Format things
\usepackage{longtable}
\usepackage{booktabs}
\usepackage{multicol}

% Math things
\usepackage{amsmath}
\usepackage{amssymb}
\usepackage{mathtools}
\usepackage{hyperref}
% ceil/floor, abs
\DeclarePairedDelimiter\ceil{\lceil}{\rceil}
\DeclarePairedDelimiter\floor{\lfloor}{\rfloor}
\DeclarePairedDelimiter\abs{\lvert}{\rvert}
% cis, exp
\DeclareMathOperator{\cis}{cis}
% nPr, nCr
\newcommand*{\permcomb}[4][0mu]{{{}_{#3}\mkern#1#2_{#4}}}
\newcommand*{\perm}[1][-3mu]{\permcomb[#1]{P}}
\newcommand*{\comb}[1][-1mu]{\permcomb[#1]{C}}
% redefine * as \times
\DeclareMathSymbol{*}{\mathbin}{symbols}{"02}

\title{2019 Challenge}
\author{James Ah Yong, QiLin Xue}
\date\today

\begin{document}

\maketitle

\section{Introduction}

The problem, as presented by Dwarka (2019), is as follows:

\begin{quotation}
  Consider the digits 2, 0, 1, and 9.
  Using each digit only once, create a formula which generates all positive integers from 1 to 100.
  Any mathematical operator is allowed, including concatenation of digits.
\end{quotation}

In this paper, we solve this problem with elementary operations and also provide a general solution using more advanced and esoteric mathematical operators.
For our purposes, an ``elementary'' operator is one of the basic infix operations: addition ($+$), subtraction ($-$), multiplication ($\times$), and division ($\div$).
Concatenation of digits (i.e.\ using 2 and 0 to form 20) and exponentiation (including radicals) are also allowed.

The ``advanced'' operations are those which would be familiar to high-school level math.
This includes: logarithms, basic and inverse trigonometric functions, and the probability functions ($x!$, $\binom{n}{r}$ or $\comb{n}{r}$, and $\perm{n}{r}$).

Finally, the ``esoteric'' level is all other operations that might aid in reducing the number of operations to find shortest solutions.
This contains functions like the floor function ($\floor{x}$) or the cis function ($\mathrm{cis}(x) = \cos(x) + i \sin(x)$).

Every solution is assigned an operational complexity $l$.
This value is defined as the number of operations required to reach the solution.
For example, $\frac{a+b}{c}=(a+b)\div c$ has a complexity of 3.
Concatenation does not count as an operation, so $20+19$ has a complexity of 1.

\section{Specific Solutions}

The following are the solutions we could find with the lowest operational complexity.
In the problem, all digits do not have to be used, however, we chose to use all digits for all problems.
Wherever possible, an effort is made to preserve the order of the digits as 2, 0, 1, 9 or its reverse.

Advanced, esoteric, and cheaty solutions are only shown if they are operationally simpler or match the 2, 0, 1, 9 order where the more elementary solution does not.

\setlength\LTleft{-1in}
\setlength\LTright{-1in}
\begin{longtable}{r@{\extracolsep{\fill}}*{3}{lr}@{}}
\toprule
$n$ & Elementary & ($l$) & Advanced & ($l$) & Esoteric & ($l$) \\ \midrule
\endhead%
0 & $0*219$ & 1 \\ \midrule
1 & $20-19$ & 1 \\ \midrule
2 & $2+0^{19}$ & 2 & & & $\floor{2.019}$ & 1 \\ \midrule
3 & $12-09$ & 1 \\ \midrule
4 & $\sqrt{9}+1^{20}$ & 3 \\ \midrule
5 & $2.0+1+\sqrt{9}$ & 3 \\ \midrule
6 & $9-2.0-1$ & 2 \\ \midrule
7 & $9-2.0*1$ & 2 \\ \midrule
8 & $9-1.0^{2}$ & 2 \\ \midrule
9 & $9+0^{12}$ & 2 \\ \midrule
\midrule
10 & $9+1.0^{2}$ & 2 \\ \midrule
11 & $2.0^{1}+9$ & 2 \\ \midrule
12 & $21-09$ & 1 \\ \midrule
13 & $12+9^0$ & 2 \\ \midrule
14 & $12+9^0$ & 2 \\ \midrule
15 & $12.0+\sqrt{9}$ & 2 & $(2+01)!+9$ & 3 \\ \midrule
16 & $2.0^{1+\sqrt{9}}$ & 3 \\ \midrule
17 & $2.0*9-1$ & 2& \\ \midrule
18 & $2.0*1*9$ & 2 \\ \midrule
19 & $29-10$ & 1 & & & $\frac{\Gamma(20)}{\Gamma(19)}$ & 3\\ \midrule
\midrule
20 & $2.0(1+9)$ & 2 & $\ceil{19.20}$ & 1 \\ \midrule
21 & $12+09$ & 1 \\ \midrule
22 & & & $2+0!+19$ & 3 & $\ceil{21.90}$ & 1 \\ \midrule
23 & $20+1\sqrt{9}$ & 3 & & & $\ceil{201\div9}$ & 2 \\ \midrule
24 & $20+1+\sqrt{9}$ & 3 & $(2+0!)1\sqrt{9}$ & 5 \\ \midrule
25 & & & ${(0!+1+\sqrt{9})}^{2}$ & 5 \\ \midrule
26 & & \\ \midrule
27 & & & $29-0!-1$ & 3 \\ \midrule
28 & $29-1.0$ & 1 \\ \midrule
29 & $20^1+9$ & 2 & & & $\floor{29.10}$ & 1\\ \midrule
\midrule
30 & $20+1+9$ & 2 \\ \midrule
31 & & & $21+0!+9$ & 3 \\ \midrule
32 & & \\ \midrule
33 & & & $(2+0!+1)!+9$ & 5 & $\sqrt{9}(10+\Gamma(2))$ & 4 \\ \midrule
34 & & & & & $\floor{\exp(\tan10)*9*2}$ & 5 \\ \midrule
35 & & & & & $\ceil{\exp(\tan10)*9*2}$ & 5  \\ \midrule
36 & & & $2(19-0!)$ & 3 \\ \midrule
37 & & \\ \midrule
38 & $2.0*19$ & 1 \\ \midrule
39 & $20+19$ & 1 \\ \midrule
\midrule
40 & & & $2(0!+19)$ & 3 \\ \midrule
41 & & \\ \midrule
42 & & \\ \midrule
43 & & \\ \midrule
44 & & \\ \midrule
45 & & \\ \midrule
46 & & \\ \midrule
47 & & & & & $\floor{\ln 201^9}$ & 3 \\ \midrule
48 & & & & & $\ceil{\ln 201^9}$ & 3 \\ \midrule
49 & & & ${(9-1-0!)}^2$ & 4 \\ \midrule
\midrule
51 & & & & & $\floor{2^9\div10}$ & 3 \\ \midrule
52 & & & & & $\ceil{2^9\div10}$ & 3 \\ \midrule
53 & & \\ \midrule
54 & & & $(2.0+1)!*9$ & 3 & $\floor{109\div2}$ & 2 \\ \midrule
55 & & & & & $\ceil{109\div2}$ & 2 \\ \midrule
56 & & \\ \midrule
57 & & \\ \midrule
58 & & & $29(0!+1)$ & 3 & $\floor{\ln 19^{20}}$ & 3 \\ \midrule
59 & & & & & $\ceil{\ln 19^{20}}$ & 3 \\ \midrule
\midrule
60 & & & $(\sqrt{9})!!\div12.0$ & 4 \\ \midrule
61 & & \\ \midrule
62 & & \\ \midrule
63 & $21.0\sqrt{9}$ & 2 \\ \midrule
64 & & & $2^{(0!+1)\sqrt{9}}$ & 5 \\ \midrule
65 & & \\ \midrule
66 & & \\ \midrule
67 & & \\ \midrule
68 & & \\ \midrule
69 & $90-21$ & 1 \\ \midrule
\midrule
70 & & \\ \midrule
71 & $91-20$ & 1 \\ \midrule
72 & & & $\perm{9^1}{02}$ & 2 \\ \midrule
73 & & & $\perm{9}{2}+1.0$ & 2 \\ \midrule
74 & & & $\perm{9}{2}+1+0!$ & 3\\ \midrule
75 & & \\ \midrule
76 & & & \\ \midrule
77 & & \\ \midrule
78 & $90-12$ & 1 \\ \midrule
79 & & & $9^2-0!-1$ & 4 \\ \midrule
\midrule
80 & $20(1+\sqrt{9})$ & 3 \\ \midrule
81 & & \\ \midrule
82 & $92-10$ & 1 \\ \midrule
83 & & \\ \midrule
84 & & & $90-(2+1)!$ & 3 \\ \midrule
85 & & \\ \midrule
86 & & \\ \midrule
87 & $90-1-2$ & 2 \\ \midrule
88 & & \\ \midrule
89 & $91.0-2$ & 1 \\ \midrule
\midrule
90 & & & $92-0!-1$ & 3 \\ \midrule
91 & $92.0-1$ & 1 \\ \midrule
92 & & & & & $\ceil{91.02}$ & 1 \\ \midrule
93 & $91.0+2$ & 1 \\ \midrule
94 & & \\ \midrule
95 & $190\div2$ & 1 \\ \midrule
96 & & \\ \midrule
97 & & \\ \midrule
98 & & \\ \midrule
99 & & & $9(12-0!)$ & 3 & $9(10+\Gamma(2))$ & 3  \\ \midrule
\midrule
100 & ${(9+1.0)}^{2}$ & 2 \\ \bottomrule
\end{longtable}

\section{General Formulae}

All the general formulae below make use of ``esoteric'' level mathematical operations and functions.

\subsection{Creating \emph{e}}
Throughout this section, we will use the combination of $\ln(x)$ and $e$ to create numbers.
$e$ can be created from each of the four digits with the function $\exp(x)=e^x$.
$\exp(1)=e$, so any digit can be first transformed to 1 and then to $e$.

Alternatively, if exp is considered cheating, the complex function $\cis(\theta)=\cos(\theta)+i\sin(\theta)$ along with Euler's identity can be used. 
Once again, any digit can be transformed to 1, then i, then $\cis(-i)=e^{i(-i)}=e$.

\subsection{Integers}

The general solution for we present and extend below is adapted from Wang (2019).

It is inspired by the famous four fours problem, which states that any positive whole number can be created with just four 4s, relying on the use of square roots:

\begin{equation}
  n = -\log_\frac{\sqrt{4}}{4}\left( \log_4 \underbrace{\sqrt{\sqrt{\cdots\sqrt{\sqrt{4}}}}}_{n} \right)
\end{equation}

where $n=\textrm{number of radicals}$.
We can extend this to the general case by recognizing that the only requirement is that the base of the outer logarithm has to be two and the base of the inner logarithm has to be the same as the radicand.
We can prove this:

\begin{align*}
  n &= -\log_2\left(\log_a \underbrace{\sqrt{\sqrt{\cdots\sqrt{\sqrt{a}}}}}_{k} \right) \\
  &= -\log_2\left(\log_a \left(a^{0.5^k}\right)\right) \\
  &= -\log_2\left(\frac{1}{2^k}\right) \\
  &= -(-k) \\
  &= k \\
\end{align*}

We can further simplify this expression by switching into base $e$.
We can use this to write any integer with only two numbers:
\begin{align*}
  n &= -\log_2\left( \ln \sqrt{\cdots\sqrt{e}}\right) \\
  &= -\log_2\left( \ln \sqrt{\cdots\sqrt{\exp1}}\right)
\end{align*}

This formula holds for all positive integers $n\in\mathbb{Z}^+$.
Zero can be created with no square roots.
By removing the leading negative, we can create all integers $n\in\mathbb{Z}$.
To enforce the use of all four digits, simply add a $+0^9$ term to the end.

\begin{equation}
  n = -\log_2\left(\ln \sqrt{\cdots\sqrt{\exp1}}\right)+0^9
\end{equation}

The operational complexity for this formula is $l=n+6$ for $n>0$ and $l=n+5$ for $n\leq0$.

\subsection{Rationals}
Since we can represent any integer with two numbers, we can represent any rational number with two sets of two numbers.
For any $n=\frac{a}{b}$, where $n\in\mathbb{Q}$:
\begin{align*}
  n &= \frac{a}{b} \\
  &= \frac{
      \log_2\left(
        \ln \underbrace{\sqrt{\cdots\sqrt{\exp1}}}_{a} 
      \right)
    }{
      \log_2\left(
        \ln \underbrace{\sqrt{\cdots\sqrt{\exp1}}}_{b} 
      \right)
    } \\
  &= \log_{
      \ln \underbrace{\sqrt{\cdots\sqrt{\exp1}}}_{b}
    }
    \left(
      \ln \underbrace{\sqrt{\cdots\sqrt{\exp1}}}_{a}
    \right)
\end{align*}

Once again, applying to 2019:
\begin{equation}
  n = \log_{
      \ln\sqrt{\cdots\sqrt{\exp(2^0)}}
    }
    \left(
      \ln\sqrt{\cdots\sqrt{\exp(1^9)}}
    \right)
\end{equation}

Following the same logic as for integers, a negative sign can be prepended for negative rationals.
Thus, the operational complexity is $l=7+a+b$ for $n\geq0$ and $l=8+a+b$ for $n<0$.

\subsection{Complex}
\subsubsection{Gaussian Integers}
Very similar to how we created the set of rationals, all Gaussian integers $\mathbb{Z}[i]=\{a+bi\mid a,b\in\mathbb{Z}\}$ can created as the sum of integers. For any $n\in\mathbb{Z}[i]$:
\begin{align*}
  n &= a + bi \\
  &= -\log_2\left(
      \ln \underbrace{\sqrt{\cdots\sqrt{\exp1}}}_{a} 
    \right) -
    \log_2\left(
      \ln \underbrace{\sqrt{\cdots\sqrt{\exp1}}}_{b} 
    \right)i \\
  &= -\log_2\left(
      \ln \sqrt{\cdots\sqrt{\exp1}}
      \times
      \ln^i \left(\sqrt{\cdots\sqrt{\exp1}}\right)
    \right) \\
  &= -\log_2\left(
      \ln \sqrt{\cdots\sqrt{\exp1}}
      \times
      \ln^{\sqrt{-1}} \left(\sqrt{\cdots\sqrt{\exp1}}\right)
    \right)
\end{align*}

Applying to 2019:
\begin{equation}
  n = -\log_2\left(
    \ln \sqrt{\cdots\sqrt{\exp0!}}
    \times
    \ln^{\sqrt{-1}} \left(
      \sqrt{\cdots\sqrt{\exp .\overline{9}}}
    \right)
  \right)
\end{equation}

The above equation has a operational complexity of $l=13+a+b$.
This works for all Gaussian integers with positive $a$ and $b$.
Apply the complex conjugate operator $\bar{n}$ to create $a-bi$ ($l=14+a+b$).
Remove the leading negative to create numbers of the form $-a-bi$ ($l=12+a+b$).
Finally, for those of the form $-a+bi$, do both ($l=13+a+b$).

\subsubsection{Gaussian Rationals}
Real rationals, like real integers, can be created with just two numbers.
From this, we can create the set of Gaussian rationals $\mathbb{Q}[i]=\{\frac{a}{b}\mid a,b\in\mathbb{Z}[i]\}$.
% TODO: prove this
Every $n\in\mathbb{Q}[i]$ can be expressed as the sum $q_1+q_2i$ where $q_1,q_2\in\mathbb{Q}$.
\begin{align*}
  n &= q_1+q_2i \\
  &= \frac{a_1}{b_1} + \frac{a_2}{b_2}i
\end{align*}
where $a_1, a_2, b_1, b_2\in\mathbb{Z}$.
It is important that our fractions contain the same denominator.
Let $d=\textrm{lcm}(b_1, b_2)$, $k_1=\frac{d}{a_1}$, $k_2=\frac{d}{a_2}$.
\begin{align*}
  &= \frac{k_1}{d} + \frac{k_2}{d}i \\
  &= \log_{
      \ln \underbrace{\sqrt{\cdots\sqrt{\exp1}}}_{d}
    }
    \left(
      \ln \underbrace{\sqrt{\cdots\sqrt{\exp1}}}_{k_1}
    \right) +
    \log_{
      \ln \underbrace{\sqrt{\cdots\sqrt{\exp1}}}_{d}
    }
    \left(
      \ln \underbrace{\sqrt{\cdots\sqrt{\exp1}}}_{k_2}
    \right)i \\
  &= \log_{
      \ln \underbrace{\sqrt{\cdots\sqrt{\exp1}}}_{d}
    }
    \left(
      \ln \underbrace{\sqrt{\cdots\sqrt{\exp1}}}_{k_1}
      \times
      \ln^{i} \underbrace{\sqrt{\cdots\sqrt{\exp1}}}_{k_2}
    \right)
\end{align*}

Substituting in the digits of 2019:
\begin{equation}
  n = \log_{\ln \sqrt{\cdots\sqrt{\exp\Gamma(2)}}}
    \left(
      \ln \sqrt{\cdots\sqrt{\exp\sqrt{0!}}}
      \times
      \ln^{\sqrt{-1}} \sqrt{\cdots\sqrt{\exp .\overline{9}}}
    \right)
\end{equation}
There are a lot of square root chains here.
The total number of variadic square roots in this expression is
\begin{equation*}
  n_{sqrt}=d+k_1+k_2=\textrm{lcm}(b_1,b_2) * \left(1+\frac{1}{a_1}+\frac{1}{a_2}\right),
\end{equation*}
making the operational complexity $l=13+n_{sqrt}$.
Like the last equation, the three other sign combinations can be created with the complex conjugate operator and/or a leading negative.


\end{document}